\ShortTitle{2 Jn.}\BookTitle{2 Jean}\BFont
\noindent\hrulefill
{\footnotesize
\textit{
\bigskip
{\centering{}
\\Auteur~: Jean
\\(Gr.~: Ioannes / Origine héb.~: Yohanan)
\\Signification~: Yahweh a fait grâce
\\Thème~: Amour et vérité
\\Date de rédaction~: Env. 85 ap. J.-C.\\}
}
\textit{
\\Il semblerait que cette épître était adressée à une église se réunissant chez une personne du nom de Kyria. Jean les invite à demeurer dans la communion avec Dieu et les met en garde contre les hérésies et la fréquentation des faux docteurs.\bigskip
}
}
\par\nobreak\noindent\hrulefill
\begin{multicols}{2}
\Chap{1}
\TextTitle{Introduction}
\VerseOne{}L'ancien, à Kyria l'élue et à ses enfants, lesquels j'aime sincèrement, et que je n'aime pas moi seul, mais aussi tous ceux qui ont connu la vérité.
\VS{2}A cause de la vérité qui demeure en nous, et qui sera avec nous éternellement:
\VS{3}Que la grâce, la miséricorde et la paix de la part de Dieu le Père, et de la part du Seigneur Jésus-Christ, le Fils du Père, soient avec vous dans la vérité et dans la charité~!
\TextTitle{La marche dans la vérité et dans la charité}
\VS{4}Je me suis fort réjoui d'avoir trouvé quelques-uns de tes enfants qui marchent dans la vérité, selon le commandement que nous avons reçu du Père.
\VS{5}Et maintenant, ô Kyria~! je te prie, non comme t'écrivant un nouveau commandement, mais celui que nous avons eu dès le commencement, que nous ayons de la charité les uns pour les autres.
\VS{6}Et c'est ici la charité, que nous marchions selon ses commandements. Et c'est là son commandement, comme vous l'avez entendu dès le commencement, afin que vous l'observiez.
\TextTitle{Avertissements}
\VS{7}Car plusieurs séducteurs sont venus dans le monde, qui ne confessent pas que Jésus-Christ est venu en chair~; un tel homme est un séducteur et un antichrist.
\VS{8}Prenez garde à vous-mêmes, afin que vous ne perdiez pas le fruit du travail que vous avez fait, mais que vous en receviez une pleine récompense.
\VS{9}Quiconque transgresse la doctrine de Jésus-Christ et ne lui demeure pas fidèle n'a pas Dieu~; celui qui demeure dans la doctrine de Christ a le Père et le Fils.
\TextTitle{La séparation d'avec les mauvaises compagnies}
\VS{10}Si quelqu'un vient à vous, et qu'il n'apporte pas cette doctrine, ne le recevez pas dans votre maison, et ne le saluez pas~!
\VS{11}Car celui qui le salue participe à ses mauvaises œuvres.
\TextTitle{Conclusion}
\VS{12}Quoique j'aie plusieurs choses à vous écrire, je n'ai pas voulu les écrire avec du papier et de l'encre, mais j'espère aller vers vous, et vous parler bouche à bouche, afin que notre joie soit parfaite.
\VS{13}Les enfants de ta sœur élue te saluent. Amen~!
\PPE{}
\end{multicols}
